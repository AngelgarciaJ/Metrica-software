\documentclass{article}
\usepackage[utf8]{inputenc}
\usepackage{geometry}
\usepackage{fancyhdr}
\usepackage{amsmath}
\usepackage[backend=biber,style=alphabetic,citestyle=authoryear]{biblatex}
\addbibresource{referencias.bib}

\geometry{a4paper, margin=1in}

\pagestyle{fancy}
\fancyhf{}
\fancyhead[L]{Garcia J. Angel R.}
\fancyhead[R]{Curso de Ingeniería de Software I}
\renewcommand{\headrulewidth}{0.4pt}

\title{Consumo de Recursos}
\author{}
\date{}

\begin{document}

\maketitle

\section{Concepto}
El consumo de recursos en el contexto del desarrollo de software se refiere a la cantidad de recursos del sistema (como CPU, memoria, disco y red) que un programa utiliza mientras se ejecuta. Esta métrica es crucial para evaluar el rendimiento y la eficiencia del software. \cite{doe2018resource}

\subsection{Historia}
El interés en medir el consumo de recursos se remonta a los primeros días de la informática, cuando los recursos eran limitados y costosos. A medida que la tecnología avanzaba, la capacidad de medir y optimizar el uso de recursos se convirtió en una parte integral del desarrollo de software. \cite{fairley2014measuring}.

\subsection{Objetivos del Proceso de Medición}
\begin{itemize}
    \item Evaluar la eficiencia del software en términos de uso de recursos.
    \item Identificar cuellos de botella en el rendimiento.
    \item Optimizar el software para mejorar su desempeño.
\end{itemize}

\section{Ventajas y Desventajas}

\subsection{Ventajas}
\begin{itemize}
    \item Permite identificar y corregir cuellos de botella en el rendimiento.
    \item Ayuda a optimizar el uso de recursos, reduciendo costos operativos.
    \item Mejora la experiencia del usuario final al proporcionar un software más eficiente.
\end{itemize}

\subsection{Desventajas}
\begin{itemize}
    \item La medición y el análisis pueden ser complejos y consumir tiempo.
    \item Requiere herramientas y técnicas especializadas.
    \item No siempre es posible optimizar todos los aspectos del consumo de recursos sin comprometer otras áreas del rendimiento \cite{jones2008applied}.
\end{itemize}


\section{Limitaciones}
A pesar de sus beneficios, la medición del consumo de recursos tiene algunas limitaciones inherentes. Por ejemplo, el consumo de recursos puede variar significativamente en diferentes entornos y configuraciones de hardware. Además, optimizar el uso de un tipo de recurso puede aumentar el consumo de otro, creando un balance delicado que debe ser manejado cuidadosamente.

\pagebreak
\section{Ejemplo}
Consideremos un sistema de gestión de bases de datos que necesita ser evaluado por su consumo de recursos. El objetivo es medir el uso de CPU y memoria durante una operación de consulta intensiva.

\subsection{Descripción del Sistema}
El sistema realiza consultas complejas sobre una base de datos de gran tamaño. Las métricas a medir incluyen el tiempo de CPU utilizado y la cantidad de memoria consumida durante la ejecución de estas consultas.

% calculo que se hace
\subsection{Cálculo del Consumo de Recursos}
Para medir el consumo de recursos durante las operaciones de consulta, se utilizaron herramientas de monitoreo del sistema que registran el uso de CPU y memoria en tiempo real. A continuación, se describe el procedimiento utilizado:

\begin{enumerate}
    \item \textbf{Configuración del Entorno}: El sistema de gestión de bases de datos fue configurado en un servidor con especificaciones conocidas (e.g., CPU de 4 núcleos, 16 GB de RAM).
    \item \textbf{Ejecución de Consultas}: Se ejecutaron las consultas de prueba de manera secuencial, asegurándose de que el sistema estuviera en un estado estable antes de cada ejecución.
    \item \textbf{Monitoreo del Sistema}: Durante la ejecución de cada consulta, se utilizó una herramienta de monitoreo (como \textit{top} en Linux o \textit{Performance Monitor} en Windows) para registrar el uso de CPU y memoria.
    \item \textbf{Registro de Datos}: Los valores de consumo de CPU (en segundos) y memoria (en MB) fueron registrados para cada consulta.
\end{enumerate}

Los resultados obtenidos fueron los siguientes:

\begin{table}[h!]
    \centering
    \begin{tabular}{|c|c|c|}
        \hline
        Operación & CPU (segundos) & Memoria (MB) \\
        \hline
        Consulta 1 & 3.5 & 150 \\
        Consulta 2 & 4.0 & 170 \\
        Consulta 3 & 3.8 & 160 \\
        \hline
    \end{tabular}
    \caption{Consumo de Recursos durante Operaciones de Consulta}
\end{table}

Estos valores reflejan el tiempo total de CPU utilizado y la cantidad máxima de memoria ocupada durante la ejecución de cada consulta. El monitoreo continuo y la recolección precisa de datos aseguraron la fiabilidad de estos resultados.


\subsection{Resultados de la Medición}
A continuación, se presentan los resultados de la medición del consumo de recursos:

\begin{table}[h!]
    \centering
    \begin{tabular}{|c|c|c|}
        \hline
        Operación & CPU (segundos) & Memoria (MB) \\
        \hline
        Consulta 1 & 3.5 & 150 \\
        Consulta 2 & 4.0 & 170 \\
        Consulta 3 & 3.8 & 160 \\
        \hline
    \end{tabular}
    \caption{Consumo de Recursos durante Operaciones de Consulta}
\end{table}





\printbibliography

\end{document}
